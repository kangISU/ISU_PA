\documentclass{article}
\usepackage{lmodern}
\usepackage{algorithm}
% \usepackage{algorithmic}
% \usepackage{algorithmicx}
\usepackage{algpseudocode}
\renewcommand{\algorithmicrequire}{\textbf{Input: }}
\algnewcommand\And{\textbf{and }}
\algnewcommand\Or{\textbf{or }}
\algnewcommand\SP{\textbf{SP}}
\algnewcommand\SV{\textbf{SV}}
\algnewcommand\enqueue{\textbf{enqueue}}
\algnewcommand\joincrowd{\textbf{joincrowd}}
% \renewcommand{\algorithmicensure}{\textbf{Output: }}
% \date{\today}
\title{CS552 Assignment 2}
\author{Yuepei Li}

\begin{document}
\maketitle

\section{Question 1}

\section{Question 2}

\subsection{Version 1}

\begin{algorithm}
  \caption{Cross Bridge, Version 1, PV}
  \algorithmicrequire \textbf{int} pass[2] = \{0, 0\} \\
  \algorithmicrequire \textbf{semaphore} mutex[2] = \{1, 1\}, mutexBridge = 1
  \begin{algorithmic}
    \Procedure{Cross(\textit{i})}{}
      \State \textbf{P}(mutex[i])
      \If{pass[i] == 0}
        \State pass[i] += 1
        \State \textbf{P}(mutexBridge)
      \Else
        \State pass[i] += 1
      \EndIf
      \State \textbf{V}(mutex[i])
      \\
      \State \textbf{Cross Bridge}
      \\
      \State \textbf{P}(mutex[i])
      \State pass[i] -= 1
      \If{pass[i] == 0}
        \State \textbf{V}(mutexBridge)
      \EndIf
      \State \textbf{V}(mutex[i])
    \EndProcedure
  \end{algorithmic}
\end{algorithm}

\begin{algorithm}
  \caption{Cross Bridge, Version 1, Monitor}
  \algorithmicrequire \textbf{int} pass[2] = \{0, 0\} \\
  \algorithmicrequire \textbf{condition} OKtoPass[2] \\
  % \algorithmicrequire \textbf{boolean} busy = false

  \begin{algorithmic}
    \Procedure{startcross(\textit{i})}{}
      \If{pass[1-i] $>$ 0}
        \State OKtoPass[i].wait
      \EndIf
      \State pass[i] += 1
      \State OKtoPass[i].signal
    \EndProcedure
    \\
    \Procedure{endcross(\textit{i})}{}
      \State pass[i] -= 1
      \If{pass[i] == 0}
        \State OKtoPass[1-i].signal
      \EndIf
    \EndProcedure
  \end{algorithmic}
\end{algorithm}

\subsection{Version 2}

\begin{algorithm}
  \caption{Cross Bridge, Version 2, Simultaneous P/V}
  \algorithmicrequire \textbf{semaphore} n[2] = \{N\} \Comment{N is the capability of the bridge}

  \begin{algorithmic}
    \Procedure{Cross(\textit{i = 0})}{}
      \State \SP(n[i], 1, 1)
      \State \SP(n[1-i], N, 0)
      \State CrossBridge
      \State \SV(n[i], 1, 1)
    \EndProcedure

    \Procedure{Cross(\textit{i = 1})}{}
      \State \SP(n[1-i], N, 0)
      \State \SP(n[i], 1, 1)
      \State CrossBridge
      \State \SV(n[i], 1, 1)
    \EndProcedure
  \end{algorithmic}
\end{algorithm}


\begin{algorithm}
  \caption{Cross Bridge, Version 2, Monitor}
  \algorithmicrequire \textbf{int} pass[2] = \{0, 0\} \\
  \algorithmicrequire \textbf{condition} OKtoPass[2] \\
  % \algorithmicrequire \textbf{boolean} busy = false

  \begin{algorithmic}
    \Procedure{startcross(\textit{i})}{}
      \If{pass[1-i] $>$ 0}
        \State OKtoPass[i].wait
      \EndIf
      \If{\textit{i} == 1 \And OKtoPass[1-i].queue}
        \State OKtoPass[i].wait
      \EndIf
      \State pass[i] += 1
      \State OKtoPass[i].signal
    \EndProcedure
    \\
    \Procedure{endcross(\textit{i})}{}
      \State pass[i] -= 1
      \If{pass[i] == 0}
        \State OKtoPass[1-i].signal
      \EndIf
    \EndProcedure
  \end{algorithmic}
\end{algorithm}

\subsection{Version 3}

\begin{algorithm}
  \caption{Cross Bridge, Version 3, Monitor}
  \algorithmicrequire \textbf{int} pass[2] = \{0, 0\} \\
  \algorithmicrequire \textbf{condition} OKtoPass[2] \\
  % \algorithmicrequire \textbf{boolean} busy = false

  \begin{algorithmic}
    \Procedure{startcross(\textit{i})}{}
      \If{OKtoPass[1-i].queue \Or pass[1-i] $>$ 0}
        \State OKtoPass[i].wait
      \EndIf
      \State pass[i] += 1
      \If{!OKtoPass[1-i].queue}
        \State OKtoPass[i].signal
      \EndIf
    \EndProcedure
    \\
    \Procedure{endcross(\textit{i})}{}
      \State pass[i] -= 1
      \If{pass[i] == 0}
        \State OKtoPass[1-i].signal
      \EndIf
    \EndProcedure
  \end{algorithmic}
\end{algorithm}


\begin{algorithm}
  \caption{Cross Bridge, Version 3, Serializer}
  \algorithmicrequire \textbf{queue} q[2] \\
  \algorithmicrequire \textbf{crowd} crowd[2] \\

  \begin{algorithmic}
    \Procedure{Cross(\textit{i})}{}
      \State \enqueue(q[i]) until ((empty(crowd[i]) \And empty(crowd[1-i])) \\
      \Comment{no one on the bridge}
      \State \textit{           } \Or (empty(q[1-i]) \And !empty(crowd[i]))) \\
      \Comment{flow car on same direction}
      \State \joincrowd(crowd[i])
        \State \textit{          }cross bridge
      \State \textbf{end}
    \EndProcedure
  \end{algorithmic}
\end{algorithm}


% -------------------------

\begin{algorithm}
\caption{Euclid’s algorithm}\label{euclid}
\begin{algorithmic}[1]
\Procedure{Euclid}{$a,b$}\Comment{The g.c.d. of a and b}
\State $r\gets a\bmod b$
\While{$r\not=0$}\Comment{We have the answer if r is 0}
\State $a\gets b$
\State $b\gets r$
\State $r\gets a\bmod b$
\EndWhile\label{euclidendwhile}
\State \textbf{return} $b$\Comment{The gcd is b}
\EndProcedure
\end{algorithmic}
\end{algorithm}



This is my \emph{first} document prepared in \LaTeX.
\end{document}
